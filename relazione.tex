\documentclass[11pt, a4paper, titlepage, block]{article}
\usepackage{listings}
\usepackage{graphicx}
\hyphenpenalty=10000
\usepackage[utf8]{inputenc}
\usepackage{pdfpages}
\begin{document}
	\begin{titlepage}

		\newcommand{\HRule}{\rule{\linewidth}{0.5mm}} % Defines a new command for the horizontal lines, change thickness here

		\center % Center everything on the page

		%----------------------------------------------------------------------------------------
		%  HEADING SECTIONS
		%----------------------------------------------------------------------------------------

		\textsc{\LARGE Universit\`a di Urbino}\\[1.5cm] % Name of your university/college
		\textsc{\Large Informatica Applicata}\\[0.5cm] % Major heading such as course name
		\textsc{\large Programmazione Procedurale e Logica}\\[0.5cm] % Minor heading such as course title

		%----------------------------------------------------------------------------------------
		%  TITLE SECTION
		%----------------------------------------------------------------------------------------


		\HRule \\[0.4cm]
		{ \huge \bfseries Relazione}\\[0.2cm] % Title of your document
		\HRule \\[0.4cm]
		\textsc{\large Progetto per la sessione invernale 2014/2015}
		\\[2cm]
		%----------------------------------------------------------------------------------------
		%  AUTHOR SECTION
		%----------------------------------------------------------------------------------------

		\begin{minipage}{\textwidth}
			\begin{flushleft}
				\emph{Studente:}\\
				Marco \textsc{Tamagno}\\ % Your name
				matricola no: 261985
				\\[1cm]
				\emph{Studente:}\\
				Julian \textsc{Sparber}\\ % Your name
				matricola no: 260492\\
			\end{flushleft}
		\end{minipage}\\[3cm]

		\begin{minipage}{\textwidth}
			\begin{flushright}
				\emph{Professore:} \\
				 \textsc{Della Selva}\\ % Supervisor`s Name
			\end{flushright}
		\end{minipage}\\[4cm]

		{\today}\\[1cm]


		%----------------------------------------------------------------------------------------
		%  DATE SECTION
		%----------------------------------------------------------------------------------------

	 % Date, change the \today to a set date if you want to be precise
		%----------------------------------------------------------------------------------------
		%  LOGO SECTION
		%----------------------------------------------------------------------------------------
		%\includegraphics{Logo}\\[1cm] % Include a department/university logo - this will require the graphicx package
		%----------------------------------------------------------------------------------------
		\newpage
		\tableofcontents
		\newpage

	\end{titlepage}

	\section{Specifica del Problema}
Creare sfruttando webRTC una simulazione del gioco del pong a 4 giocatori dove gli utenti connessi tramite browser si scambino informazioni tramite peer to peer

	\section{Analisi del Problema}
WebRTC tramite un iceserver sfruttato per l autenticazione mette in una comunicazione peertopeer due o piu' browser.
Il gioco del pong e' composto da questi elementi:
una pallina
una racchetta (una per ogni giocatore)
una stanza (dove verra' svolta la partita)
una rete (dove verranno segnati i gol)

ogni giocatore inviera' a ogni altro giocatore la posizione della propia racchetta.
il giocatore master, in possesso della pallina comunichera' a tutti gli altri giocatori la posizione della pallina
ogni giocatore inviera' agli altri giocatori il numero di reti che ha subito.

Ogni stanza ospitera' una partita differente.
Gli utenti "extra" potranno far da spettatori alla partita.

\end{document}
